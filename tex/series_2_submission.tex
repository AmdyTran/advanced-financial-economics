% Advanced Financial Economics - Series 2
% Author: Andy Tran
\documentclass[11pt,a4paper]{article}

% Packages
\usepackage[utf8]{inputenc}
\usepackage[T1]{fontenc}
\usepackage{amsmath,amssymb,amsfonts}
\usepackage{mathtools}
\usepackage{graphicx}
\usepackage{xcolor}
\usepackage{hyperref}
\usepackage{geometry}
\usepackage{fancyhdr}
\usepackage{titlesec}
\usepackage{enumitem}
\usepackage{booktabs}
\usepackage{microtype}
\usepackage{datetime}

% Page setup
\geometry{margin=1in}
\setlength{\parindent}{0pt}
\setlength{\parskip}{6pt}

% Colors
\definecolor{darkblue}{RGB}{0,0,102}
\definecolor{darkred}{RGB}{128,0,0}

% Hyperref setup
\hypersetup{
    colorlinks=true,
    linkcolor=darkblue,
    filecolor=darkblue,
    urlcolor=darkblue,
    citecolor=darkblue
}

% Header and footer
\pagestyle{fancy}
\fancyhf{}
\fancyhead[L]{Andy Tran}
\fancyhead[R]{Advanced Financial Economics}
\fancyfoot[C]{\thepage}
\renewcommand{\headrulewidth}{0.4pt}
\renewcommand{\footrulewidth}{0.4pt}

% Title format
\titleformat{\section}
  {\normalfont\Large\bfseries\color{darkblue}}
  {\thesection}{1em}{}
\titleformat{\subsection}
  {\normalfont\large\bfseries\color{darkred}}
  {\thesubsection}{1em}{}

\begin{document}

\begin{center}
    \LARGE{\textbf{Advanced Financial Economics}}\\[0.5cm]
    \Large{\textbf{Series 2}}\\[0.5cm]
    \large{Andy Tran}\\[0.2cm]
    \large{\today}
\end{center}

\section*{Exercise 1}
We know that $\beta=0.5$ and $H=2$, $(e_1^1, e_2^1) = (1,0)$ and $(e_1^2, e_2^2) = (0,1)$.

\subsection*{a) CARA Utilities}
From the lecture we know that we have to solve
$$ \log(1+r) + \log(\beta) = \frac{1}{H} \sum_{i=1}^{H} (e_2^h - e_1^h)$$

Using the given values we get
$$ \log(1+r) + \log(0.5) = \frac{1}{2} ((0 - 1) + (1-0)) = 0 \Rightarrow \log(1+r) = -\log(0.5) = \log(2)$$
This yields $r=1$.

\subsection*{b) CRRA Utilities}
From the lecture we know that we have to solve
$$ \beta(1+r) = \left(\frac{\sum_h e_2^h}{\sum_h e_1^h}\right)^\sigma$$

Using the given values we get
$$ 0.5(1+r) = \left(\frac{1}{1}\right)^2 = 1 \Rightarrow 1+r = \frac{1}{0.5} = 2 \Rightarrow r=1$$

\subsection*{c) Quadratic Utilities}
As both utility functions are the same, the marginal utility function will have to satisfy 
\begin{align*}
&\beta(1+r) = \frac{1 - \frac{1}{5}c_1}{1-\frac{1}{5}c_2}
\end{align*}

We can use that $\sum_h s^h = \sum_h (e_1^h - c_1^h) = 0$, or equivalently
$\sum_h e_1^h = \sum_h c_1^h$ and note that $\sum_h c_2^h = \sum_h e_2^h + (1+r) \sum_h s^h \Rightarrow \sum_h c_2^h = \sum_h e_2^h$, due to market clearing.
\begin{align*}
&(1-\frac{1}{5}c_1) = \beta(1+r) (1-\frac{1}{5}c_2)
\end{align*}

Summing over each household gives us:
\begin{align*}
&\sum_h (1-\frac{1}{5}c_1^h) = \beta(1+r) \sum_h (1-\frac{1}{5}c_2^h) \\
&\Rightarrow 2 \left(1-\frac{1}{5}\sum_h c_1^h\right) = \beta(1+r) \cdot 2 \left(1-\frac{1}{5}\sum_h c_2^h\right) \\
&\Rightarrow 2 \left(1-\frac{1}{5}\sum_h e_1^h\right) = \beta(1+r) \cdot 2 \left(1-\frac{1}{5}\sum_h e_2^h\right)
\end{align*}

Plugging in the given values yields:
\begin{align*}
&2 \left(1-\frac{1}{5}\right) = 0.5(1+r) \cdot 2 \left(1-\frac{1}{5}\right) \\
&\Rightarrow \frac{8}{5} = 0.5(1+r) \cdot \frac{8}{5} \\
&\Rightarrow 1 = 0.5(1+r) \\
&\Rightarrow 1+r = 2
\end{align*}
which gives us $r=1$.

\subsection*{d) Some other utilities}
We have the first marginal utility 
\begin{align*}
&\frac{\frac{1}{c_1 + 0.1}}{\frac{1}{c_2 + 0.1}} = \frac{c_2 + 0.1}{c_1 + 0.1}= 0.5(1+r) \\
&c_{2}^R + 0.1 = 0.5(1+r)(c_1^R + 0.1)
\end{align*}

For the other agent we get the marginal utility:
\begin{align*}
&\frac{c_2^M}{c_1^M} = \beta (1+r)
\end{align*}
which is equivalent to:
\begin{align*}
&c_{2}^M = 0.5 (1+r)c_{1}^M
\end{align*}

Adding them both up we get
\begin{align*}
&c_{2}^R + 0.1 + c_{2}^M + = 0.5(1+r)(c_1^R + 0.1 + c_1^M )
\end{align*}
we can use that $c_{2}^R + c_{2}^M = 1, c_{1}^R + c_{1}^M = 1$,
\begin{align*}
&1 + 0.1 = 0.5(1+r)(1+ 0.1) \\
&\Rightarrow 1.1 = 0.5(1+r)(1.1) \\
&\Rightarrow r = 1
\end{align*}

\subsection*{e) Another utility}
The first marginal utility gives us:
\begin{align*}
&c_2 = (1+r) \\
&\Rightarrow s^R(1+r) + e_2 = (1+r) \\
&\Rightarrow s^R(1+r) + 0 = (1+r)
\end{align*}

Assuming that $(1+r) \neq 0$, we retrieve $s^R = 1$

The second marginal utility gives us:
\begin{align*}
&c_2 = 0.5(1+r) \\
&\Rightarrow s^M(1+r) + e_2 = 0.5(1+r)
\end{align*}
and using that $s^M = -s^R = -1$ we get:
\begin{align*}
&-1(1+r) + 1 = 0.5(1+r) \\
&\Rightarrow -(1+r) + 1 = 0.5(1+r) \\
&\Rightarrow -1 - r + 1 = 0.5 + 0.5r \\
&\Rightarrow -r = 0.5 + 0.5r \\
&\Rightarrow -1.5r = 0.5 \\
&\Rightarrow r = -\frac{1}{3}
\end{align*}

\subsection*{f) Last utility}
The first margin utility gives us:
\begin{align*}
&\frac{c_2^R}{c_1^R} = 1+r \\
&\Rightarrow \frac{s^R(1+r) + e_2^R}{e_1^R - s^R} = 1+r \\
&\Rightarrow \frac{s^R(1+r) + 0}{1 - s^R} = 1+r \\
&\Rightarrow \frac{s^R(1+r)}{1 - s^R} = 1+r \\
&\Rightarrow s^R(1+r) = (1+r)(1 - s^R) \\
&\Rightarrow s^R(1+r) = (1+r) - s^R(1+r) \\
&\Rightarrow 2s^R(1+r) = (1+r) \\
&\Rightarrow 2s^R = 1 \\
&\Rightarrow s^R = 0.5
\end{align*}

The second marginal utility gives us:
\begin{align*}
&\frac{c_2^M}{c_1^M} = 0.5(1+r) \\
&\Rightarrow \frac{s^M(1+r) + e_2^M}{e_1^M - s^M} = 0.5(1+r) \\
&\Rightarrow \frac{-s^R(1+r) + e_2^M}{e_1^M + s^R} = 0.5(1+r) \\
&\Rightarrow \frac{-0.5(1+r) + 1}{0 + 0.5} = 0.5(1+r) \\
&\Rightarrow \frac{-0.5(1+r) + 1}{0.5} = 0.5(1+r) \\
&\Rightarrow -1(1+r) + 2 = 0.5(1+r) \\
&\Rightarrow -1 - r + 2 = 0.5 + 0.5r \\
&\Rightarrow 1 - r = 0.5 + 0.5r \\
&\Rightarrow 0.5 = 1.5r \\
&\Rightarrow r = \frac{1}{3}
\end{align*}

\section*{Exercise 2: T-periods, H-agents}
Since the discount factor is the same, from the lecture we know that the Euler Equation is:
\begin{align*}
&\frac{e^{-c_{t-1}}}{e^{-c_t}} = \beta(1+r_t), \; \forall t\in[1,T] \\
&\Rightarrow \frac{e^{-c_{t-1}}}{e^{-c_t}} = \beta(1+r_t) \\
&\Rightarrow e^{c_t - c_{t-1}} = \beta(1+r_t) \\
&\Rightarrow c_{t} - c_{t-1} = \log(\beta(1+r_t)) \\
&(e_t^h + s_{t-1}^h(1+r_t)) - (e_{t-1}^h + s_{t-2}^h(1+r_{t-1}) - s_{T-1}) = \log(\beta(1+r))
\end{align*}

If we sum up over all agents, and using the condition that the markets clear, we easily see that:
\begin{align*}
&\sum_h e_t^h + 0 - \sum_h e_{t-1}^h + 0 - 0 = H \log(\beta(1+r_t)) \\
&\Rightarrow \sum_h (e_t^h - e_{t-1}^h) = H \log(\beta(1+r_t)) \\
&\Rightarrow \sum_h 1 = H \log(\beta(1+r_t)) \\
&\Rightarrow H = H \log(\beta(1+r_t)) \\
&\Rightarrow 1 = \log(\beta(1+r_t)) \\
&\Rightarrow 1 - \log(\beta) = \log(1+r_t) \\
&\Rightarrow e^{1 - \log(\beta)} = 1+r_t \\
&\Rightarrow \frac{e^1}{e^{\log(\beta)}} = 1+r_t \\
&\Rightarrow \frac{e}{\beta} = 1+r_t \\
&\Rightarrow r_t = \frac{e}{\beta} - 1
\end{align*}

Thus the interest rate is always the same for each timestep we are at.

\end{document} 